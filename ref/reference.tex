\documentclass[acmsmall]{acmart}
\usepackage{hyperref}
\usepackage{amsmath, amsfonts}
\usepackage{algorithmic}
\usepackage{graphicx}
\usepackage{textcomp}
\usepackage{xcolor}
\usepackage{booktabs}
\usepackage{nicematrix}
\usepackage{catchfile}
\usepackage{pgfplots}
\usepackage{subcaption}
\usepackage{mdframed}
\usepackage{multirow}
\usepackage{arydshln}
\usepackage{colortbl}
\usepackage[most]{tcolorbox}
\usepackage{lipsum}
\usepackage{pifont}
\usepackage{makecell}
\usepackage{threeparttable}

\AtBeginDocument{ \providecommand{\BibTeX}{{ \normalfont B\kern-0.5em{\scshape i\kern-0.25em b}\kern-0.8em\TeX}}}

\settopmatter{printacmref=false} % Removes citation information below abstract
\renewcommand{\footnotetextcopyrightpermission}[1]{} % removes footnote with conference information in first column
\makeatletter
\let\@authorsaddresses\@empty
\makeatother

% 评论框样式定义
\tcbset{ commentbox/.style={ enhanced, colback=blue!5!purple!5, colframe=gray!120, coltitle=white, colbacktitle=gray!120, fonttitle=\bfseries, boxrule=0.5pt, arc=2mm, top=1mm, bottom=1mm, left=2mm, right=2mm, } }

\pretocmd{\section}{\clearpage}{}{}%

\begin{document}
	\pagestyle{plain}
	\begin{center}
		\Large Response Letter for the Manuscript \#TOSEM-2025-0008.R1 \\[1.5em]

		\LARGE\textbf{Patch Generation in APR: A Survey from the Perspectives of Utilizing LLMs and Using
		APR-Specific Information} \\[1.5em]

		\large ACM Transactions on Software Engineering and Methodology \\[1em]

		\normalsize \textit{August 6, 2025} \\[1.5em]

		\normalsize Yaopeng Yang$^{1}$, Chuanyi Li$^{1\ast}$, Zhifeng Han$^{2}$, Rui Li$^{2}$, Kui
		Xu$^{2}$, \\ Qingyuan Li$^{1}$, Wenkang Zhong$^{1}$, Zongwen Shen$^{1}$, Zhiwei Fei$^{1}$,
		Jidong Ge$^{1}$, Bin Luo$^{1}$ \\[1em]

		$^{1}$\textit{State Key Laboratory for Novel Software Technology, Nanjing University, China}
		\\

		$^{2}$\textit{China Mobile, China} \\[10em]
	\end{center}

	\renewcommand{\shortauthors}{YaoPeng Yang, et al.}
	\newcommand{\SectionPosition}[1]{\textbf{$\triangleright$ #1}}

	Firstly, we would like to thank you for your kind letter and constructive comments concerning
	our paper. All of these comments are valuable and helpful to improve our paper. We have
	thoroughly discussed all of these comments and made our best effort to update our manuscript
	according to your feedback to meet the requirements of ACM TOSEM. The positions in the revised manuscript
	are highlighted by {\color{red} \textbf{RED}} color, and the content of the main changes of the
	revised manuscript is highlighted by {\color{blue} \textbf{BLUE}} color. Point-by-point responses
	to the reviewers are listed below.

	\section*{Meta-review from the Editors}
	\begin{tcolorbox}
		[commentbox,title=Reviewer \#0 - Comment 1] Overall, the revision has addressed most of the issues
		raised by the two reviewers. However, there are some minor issues to be fixed: 1. missing
		citation to the formal publication of SWE-bench, 2. some presentation issues.
	\end{tcolorbox}

	\noindent
	\textbf{Response.}

	Dear editor, we sincerely thank you for the positive evaluation of our revised manuscript.

	Regarding the minor issues you mentioned, we have:

	(1). added a proper citation to the formal publication of SWE-bench. The reference has been
	updated in the citation. In addition, we thoroughly reviewed the citations and corrected several
	citation errors that had not been previously discovered (e.g., corrected the previous issue of references
	to SWE-bench Leaderboards without authorship).

	(2). carefully revised the manuscript to address the identified presentation issues.
	Specifically, we improved the overall clarity by using the uniform Takeaways format: Takeaways
	of RQn-[Topic].
	\vspace{1em}

	In addition, we have thoroughly revised and addressed each of the original comments from Reviewer
	\#2 that were not acknowledged in the previous round. Specifically, we have:

	(1). summarized the main contributions of the paper in the Introduction section to help readers quickly
	grasp the core insights of this survey.

	(2). removed or consolidated ambiguous columns in Table 2 to enhance clarity and
	interpretability.

	(3). enriched Figures 3 and 12 by incorporating detailed percentage information to improve their
	informativeness.

	(4). added Table 6 to systematically summarize and compare the three training strategies
	discussed in the paper, offering a clearer understanding of their respective characteristics.
	\vspace{1em}

	We sincerely hope that the revised manuscript meets the accepted standards of ACM TOSEM.

	%=======================审稿人1=========================
	\section{Comments from Reviewer \#1}
	%\#\# Summary:

	%Thank you very much for the revisions. Most of my questions have been well addressed, and the manuscript has been revised accordingly. I would like to offer the following suggestions for the new version submitted.

	\begin{tcolorbox}
		[commentbox,title=Reviewer \#1 - Comment 1] Recommendation: Accept

		Comments: (There are no comments.)
	\end{tcolorbox}

	\noindent
	\textbf{Response.} We sincerely appreciate your positive evaluation and support for our manuscript.
	We are grateful for your time and effort in reviewing our work.

	%===========================审稿人2====================

	\section{Rviewer 2}

	\begin{tcolorbox}
		[commentbox,title=Reviewer \#2 - Comment 1] For this revision, I appreciate the authors'
		revision, especially in addressing the test-time scaling and Repository-Level Repair Coverage.
		Some minor comments: 1. Ensure to cite the formal publication of SWE-bench 2. For takeaway,
		please use consistent titles: e.g. use Takeaway - add both Topic and RQ X
	\end{tcolorbox}

	\noindent
	\textbf{Response.}

	Thank you for your valuable comments regarding the citation format of SWE-bench and the presentation
	of the Takeaways.

	% 之前对SWE-bench的引用是来自arXiv的预印版,现已改为正式的publication引用

	% Carlos E Jimenez, John Yang, Alexander Wettig, Shunyu Yao, Kexin Pei, Ofir Press, and Karthik R Narasimhan. 2024. SWE-bench: Can Language Models Resolve Real-world Github Issues?. In The Twelfth International Conference on Learning Representations. https://openreview.net/forum?id=VTF8yNQM66

	% 此外,Takeaway也统一了格式:Takeaways of RQn-Topic。以下是一个例子:

	The previous citation of SWE-bench referred to its arXiv preprint version. We have now updated it
	to cite the formal publication:

	\color{blue}
	Carlos E. Jimenez, John Yang, Alexander Wettig, Shunyu Yao, Kexin Pei, Ofir Press, and Karthik R.
	Narasimhan. 2024. SWE-bench: Can Language Models Resolve Real-world GitHub Issues? In
	Proceedings of the Twelfth International Conference on Learning Representations.\url{https://openreview.net/forum?id=VTF8yNQM66}
	\color{black}

	In addition, we have standardized the format of the Takeaways across the paper. The new format follows
	the structure: Takeaways of RQn-[Topic]. Below is an example:

	\color{blue}
	Takeaways of RQ3-Repository-level Repair
	\color{black}

	\begin{tcolorbox}
		[commentbox,title=Reviewer \#2 - Comment 2] The authors have made substantial improvements
		to this survey paper in response to the initial reviews. However, there appears to have been
		some miscommunication, as the review I previously submitted does not seem to have been made available
		to the authors. (I apologize if I contributed to this.) I would appreciate it if the authors
		could address the comments below, as I believe the paper should not be published until these
		points are properly considered.
	\end{tcolorbox}

	\noindent
	\textbf{Response.}

	We regret that the earlier review was not available to us during the previous revision round. If
	it had been, we would certainly have addressed those comments at that time. We are grateful that
	you have shared them again, and we have now carefully revised the manuscript in response to each
	point raised. \textit{We would like to respectfully point out that the majority of the concerns
	you raised in your original review were already thoroughly addressed in our first-round revision.}

	\begin{tcolorbox}
		[commentbox,title=Reviewer \#2 - Comment 3] Introduction section, I recommend authors clearly
		articulate the contributions of this research in the introduction section.
	\end{tcolorbox}

	\noindent
	\textbf{Response.}

	We have added a summary of contributions in the Introduction (
	\color{red}
	Section 1
	\color{black}
	) to help readers quickly grasp the core insights of this survey. Details are as follows:

	\color{blue}
	\begin{itemize}
		\item \textbf{Four Analytical Dimensions:} We organize the discussion of LLM-based patch
			generation methods across four well-structured dimensions—Task Style, LLM Utilization Techniques,
			Reasoning Strategy, and Input Strategy—providing a layered and systematic analysis framework.

		\item \textbf{Three Key Factors:} We explore three critical factors that influence the
			performance of LLM-based APR: the selection of LLMs, the input information (including input
			representation and context), and fine-tuning configurations (including datasets and
			three categories of fine-tuning strategies).

		\item \textbf{Three Evaluation Focuses:} For the first time, we summarize the trends and characteristics
			of repository-level benchmarks and corresponding repair, exploring the test-time scaling
			effect in patch generation. Furthermore, we reveal the data leakage issues and
			mitigation strategies.

		\item \textbf{Two Complementary Perspectives:} Beyond the overarching LLM Utilization
			Perspective, we introduce a novel APR-specific Perspective that emphasizes the value and
			characteristics of APR-specific information. Through targeted case analyses, we extract three
			insightful conclusions.

		\item \textbf{Ten Future Directions:} Grounded in real-world challenges and current research
			bottlenecks, we articulate ten well-founded and forward-looking directions for the future
			development of APR in the LLM era.
	\end{itemize}
	\color{black}

	\begin{tcolorbox}
		[commentbox,title=Reviewer \#2 - Comment 3] line 28-33, Page 6: A major concern is that I'm
		not convinced that the difference between this work and Zhang et al. is clearly described. I
		feel very reluctant to say this paper adds a quantifiable contribution on top of Zhang et al.'s
		work. It's unfortunate that a number of contents mentioned in this paper have already been
		discussed in Zhang et al., such as "distributions of involved LLMs", etc. For the research
		questions, RQ1 and RQ2 have also been extensively discussed by Zhang et al., although the authors
		tried to highlight the "APR-specific information" in the discussions. Authors should modify
		the paper so that the contribution on top of Zhang et al. can be clearly articulated.
	\end{tcolorbox}

	\noindent
	\textbf{Response.}

	Thank you for raising this point. We would like to note that this issue was already addressed in
	our previous revision and response. In fact, a similar comment was raised by another reviewer in
	the earlier round, as we made substantial revisions to the manuscript accordingly and carefully
	discussed the similarities and differences between our work and that of Zhang et al. \cite{S2}.

	The following is the relevant content of the previous response letter. ($\Downarrow$)
	\vspace{1em}
	\hrule \hrule
	\vspace{1em}
	\textbf{Main Similarities:}
	\begin{itemize}
		\item \textbf{Background Overview of APR.}
			\newline
			Both provide an overview of the background of APR, e.g., the main phrases of APR, and
			the historical evolution of patch generation techniques.
			\newline

			\textbf{Analysis:} Our treatment of this content corresponds to
			\color{red}
			Section 2.1
			\color{black}
			of the revised manuscript. Such an overview is essential to orient readers---particularly
			those unfamiliar with APR---and facilitate deeper engagement with the rest of the survey.

			\hspace*{1em}Furthermore, our background description is more tightly aligned with our
			specific research focus. For example, we give a more detailed account of template-based
			techniques in the history of patch generation methods (
			\color{red}
			Page 4, Line 24
			\color{black}
			), which lays the foundation for understanding later discussions on repair templates. Conversely,
			the section on defect localization (
			\color{red}
			Page 3, Line 18
			\color{black}
			) is kept concise and contextually relevant, since our primary focus is patch generation
			using LLMs.
			\newline

		\item \textbf{Trend Analysis in APR Research.}
			\newline
			Both works analyze trends in the APR field.
			\newline

			\textbf{Analysis: }Although both analyze trends, the focus is almost entirely different.
			\textit{Review} emphasizes publication trends (e.g., by venue, programming language, and
			study type). In contrast, our Survey focuses on the substantive evolution of research content.
			Specifically, we analyze trends in inputs (e.g., APR-specific information), granularity
			of repair (from function-level to repository-level), and inference strategies (e.g.,
			iterative and agent-based approaches), providing a more technical and application-oriented
			aspect.
			\newline

		\item \textbf{Discussion of Patch Generation Techniques for LLM-based APR.}
			\newline
			Both surveys examine techniques for patch generation using LLMs.
			\newline

			\textbf{Analysis: }However, our organizational framework is significantly different from
			\textit{Review}. \textit{Review} sets Fine-tuning, Few-shot, and Zero-shot Learning (including
			cloze-style and conversational-based repair) to discuss the techniques for generating
			patches using LLM. In contrast, we propose a broader, more detailed taxonomy encompassing
			four dimensions (
			\color{red}
			Section 4.1-4.4
			\color{black}
			): Task Style, LLM Utilization Techniques, Reasoning Strategies, and Input Strategies.
			Our framework can be viewed as a superset of the \textit{Review}’s and supports a deeper
			and more comprehensive technical discussion.
			\newline
	\end{itemize}

	\textbf{Main Differences}

	\begin{itemize}
		\item \textbf{Finer-Grained and More Comprehensive Analysis of Input Information.}
			\newline
			The \textit{Review} explores the input forms of software bugs transformed into when utilizing
			LLMs, and categorizes them into Raw Bug-fixing Input, Prompt Input, Mask Input, Conversation-Style
			Representation, and Structure-Aware Input. These focus on the structure or form of the input,
			not the detailed information content. In contrast, we provide a detailed summary and
			analysis of input information according to fine-tuning and prompting techniques in
			\color{blue}
			\textbf{5.2 Input in LLM}
			\color{black}
			spanning up to 6 pages (
			\color{red}
			Section 5.2
			\color{black}
			), aiming to guide researchers in optimizing LLM inputs for patch generation.
			\newline

		\item \textbf{In-Depth Exploration of Fine-tuning Configurations.}

			The \textit{Review} briefly mentions fine-tuning in just two paragraphs in Section 5.2, whereas
			we offer a detailed discussion of the impact of different fine-tuning configurations on the
			patches generation by LLMs (
			\color{red}
			Section 5.3
			\color{black}
			). Our exploration includes fine-tuning data, training strategies. Under the latter
			aspect, we categorize related works into autoregressive fine-tuning, masked prediction
			fine-tuning, and preference alignment fine-tuning, aiming to guide researchers in
			selecting and designing effective fine-tuning methods for LLM-based APR.
			\newline

		\item \textbf{Focused Discussion on Repository-Level Repair and Benchmarks.}

			Repository-level repair has emerged as a key research trend in the recent APR community,
			yet the \textit{Review} provides minimal coverage of this topic. To our knowledge, this manuscript
			is the first survey to discuss repository-level repair systems systematically. In
			\color{red}
			Section 6.1
			\color{black}
			, we analyze \textbf{benchmark evolution} over time, highlighting the increasing
			prevalence of repository-level benchmarks (e.g., SWE-bench) over traditional function-level
			ones. Relevantly, we also analyze three unique \textbf{characteristics of repository-level
			repair} in this section and summary \textbf{pioneering SWE-bot} in
			\color{red}
			Table 6.
			\color{black}
			\newline

		\item \textbf{Valuable Reasoning Suggestions from Test-time Scaling Effect.}

			Test-time scaling---distinct from parameter scaling---has become a central topic in LLM research.
			In
			\color{red}
			Section 6.2
			\color{black}
			, we examine its implications for patch generation and introduce a taxonomy of inference
			strategies, including vertical scaling (e.g., iterative and agent methods) and parallel scaling
			(e.g., LLM-as-reviewer and decoding optimization). We present two key conclusions:

			\color{blue}
			\textbf{Conclusion 1: Within certain bounds, patch generation effectiveness improves
			with increased reasoning computation, exhibiting a trend that roughly exhibits a trend that
			roughly follows a logarithmic curve.}

			\textbf{Conclusion 2: Additional reasoning computations for smaller or weaker models can
			outperform fewer reasoning computations for stronger models. }
			\newline
			\color{black}

			and offer two practical suggestions:

			\color{blue}
			\textbf{Suggestion 1: Avoid arbitrarily, crudely extending inference length or
			introducing redundant computations.}

			\textbf{Suggestion 2: Focus on the usage and orchestration of external tools.}
			\color{black}
			\bigskip

			, which we believe are valuable contributions to the APR community, particularly as reasoning
			phrase becomes a central focus in LLM utilization and optimization.
			\newline

		\item \textbf{Reader-Oriented and Actionable Takeaways.}

			Unlike the \textit{Review}, which tends to summarize content descriptively in “Summary”,
			each item in our “Takeaways” are explicitly designed to provide actionable insights. They
			are oriented toward practitioners and researchers, aiming to facilitate better decision-making
			and foster future advancements in LLM-based APR.
	\end{itemize}

	In summary, the main relationship between our Survey's \textbf{\textit{LLM utilization
	perspective}} and the \textit{Review} is not overlapping, but rather complementary. The \textit{Review}
	excels in analyzing the literature publish trend and the breakdown of repair scenarios, which
	are not present in and are not the focus of our work. Conversely, our survey contributes deeper
	technical insights into key areas such as \textit{input optimization}, \textit{fine-tuning
	strategies}, \textit{reasoning paradigms}, \textit{repository-level repair}, and \textit{test-time
	scaling effect}.

	Moreover, it's important to note that the \textbf{\textit{APR-specific perspective}}---another core
	dimension of our survey---is completely absent from the \textit{Review}. This further highlights
	the distinctiveness and added value of our work to the LLM-based APR research community.

	\vspace{1em}
	\hrule \hrule

	\begin{tcolorbox}
		[commentbox,title=Reviewer \#2 - Comment 4] Table 2, I like the table provided by the authors
		that summarizes the differences for the surveys in this area. I appreciate it if the author
		could distinguish more clearly "APR-specific discussion" and "APR-specific / General Features".
		It would greatly help if the differences could be visualized in a graph, highlighting the
		unique contribution of this survey. On this note, I recommend authors formally formalize the
		definition of terminologies such as "APR-specific information/specifications" to help better
		understand authors' contribution (it can be neglected if it is only mentioned in the text)
	\end{tcolorbox}

	% 非常抱歉我们对Table2的展示让您感到困惑。
	% 认真分析您的回复之后,我们猜测您的疑惑首先来自于Table2中"APR-specific discussion"和""APR-specific / General Features",您认为这两个是与APR-specific information类似的两个专有名词,并且想让我们明晰这三个专有名词,最好是可视化的方式。实际上,本文的意思并非如此,接下来让我详细解释。

	% APR-specific Perspective和 APR-specific information是本文的专有名词,并且在Introduction中清晰的定义:

	% 而APR-specific discussion和APR-specific / General Features则是仅在表2中出现,并非专有名词。APR-specific discussion指的是文献综述/survey中是否出现关于APR-specific的讨论,例如 Monperrus et al. [ 141 ] propose for the first time Domain-specific Repair, i.e., doing automatic repair in contexts of a particular application domain. APR-specific / General Features指的是文献综述/survey中是否有区分"APR-specific" 和 “general”不同的特征。

	% 为了避免Table2给读者后续阅读带来的歧义,我们将"APR-specific discussion"合并到""APR-specific / General Features"。这一出发点是APR-specific / General Features其实是包括APR-specific discussion的,因为如果文献对APR-specific Features和 General Features进行对比,则必然会涉及APR-specific discussion。

	% 修改后的表二如下:

	\noindent
	\textbf{Response.}

	We sincerely apologize for the confusion caused by our presentation of
	\color{red}
	Table 2
	\color{black}
	.

	After carefully reviewing your comment, we believe the confusion may have arisen from the terms
	``APR-specific discussion'' and ``APR-specific/General Features'' in Table 2. It seems that you
	interpreted these as technical terms similar to ``APR-specific information'', and suggested that
	we clarify the distinction between these three concepts, preferably in a visualized manner.

	However, this was not our intended meaning. Please allow us to clarify.

	In our manuscript, ``APR-specific Perspective'' and ``APR-specific information'' are the two core
	concepts, and both are clearly defined in the Introduction. Detailed are as follows:

	\color{blue}
	In order to differentiate from general knowledge, we define \textbf{\textit{APR-specific
	information}} as information specifically applied to and required by APR, rather than other SE
	or general NLP tasks, and that is closely tied to bugs, including their identification and
	repair, such as error message, bug-fix pair. Given this landscape, we explore the distinctive
	role of APR-specific information in the APR task, which we term the \textbf{\textit{APR-specific
	Perspective}}. From this perspective, we emphasize that in the era of LLMs, while relying on the
	increasing general-purpose capability and extensive information inherent in models, it is
	equally crucial to focus on the mining and application of APR-specific information.
	\color{black}

	In contrast, the terms ``APR-specific discussion'' and ``APR-specific/General Features'' appear only
	in Table 2 and are not intended as technical terms.

	By ``APR-specific discussion'', we refer to whether the survey includes a discussion focused on
	APR-specific aspects. For example, Monperrus et al. \cite{S7} introduced the notion of Domain-specific
	Repair, i.e., performing automatic program repair within the context of a particular application
	domain.

	The term ``APR-specific/General Features'' indicates whether the survey explicitly distinguishes
	between APR-specific and general features in its analysis.

	To avoid potential ambiguity for readers, we have revised
	\color{red}
	Table 2
	\color{black}
	(see Figure \ref{f:t2} for detailed comparison) in the revision manuscript by merging the two entries—APR-specific
	discussion and APR-specific/General Features—into a single category. Our rationale is that any
	survey that compares APR-specific features with general features will necessarily involve an APR-specific
	discussion, and thus the latter is subsumed under the former.

	\begin{figure}[t!]
		\centering
		\includegraphics[width=1\textwidth]{fig/response-com.pdf}
		\caption{Before-and-after comparison of Table 2.}
		\label{f:t2}
	\end{figure}

	\begin{tcolorbox}
		[commentbox,title=Reviewer \#2 - Comment 5] Line 49, Page 9 (section 4): I appreciate it if
		the authors could add more details about automated fault localization, such as what percentage
		of research work needs automatic fault localization specifically.
	\end{tcolorbox}

	% 看来审稿人比较关注缺陷定位。值得指出的是,我们在上一轮修改中已经将本论文的主题划定为APR中的补丁生成,因此详细讨论缺陷定位并非本文的重点。尽管如此,实际上我们依然在上一轮的修改中添加了很多对缺陷定位的描述。这些描述被放在库级别修复的趋势探讨中(Section xx)。我们添加了很多最新的库级别缺陷定位的案例分析,详细描述了库级别缺陷定位的最新技术。我们认为这一价值已经超越了简单的百分比的统计。
	% 原文内容如下:

	\noindent
	\textbf{Response.}

	The reviewer’s concern regarding fault localization is duly noted. It is important to emphasize that,
	in the previous revision, we deliberately narrowed the scope of this paper to concentrate specifically
	on patch generation within APR. Consequently, an exhaustive treatment of fault localization falls
	outside the primary focus of this study.

	Nonetheless, we have integrated substantial material related to fault localization in our prior
	revision. In particular, we enriched the discussion on repository-level repair trends (
	\color{red}
	Section 6.1.1
	\color{black}
	) by incorporating several recent case studies that highlight advances in fault localization techniques
	at the repository level. These additions reflect the cutting-edge developments in this domain.

	We contend that this detailed qualitative analysis provides greater insight and utility than a
	mere quantitative summary based on percentages. Details in manuscripts are as follows:

	\color{blue}
	More important repository-level bug localization and repair ingredient code extraction. In
	repository-level repair, bug localization is often intertwined with repair ingredient code research
	and extraction. Specifically, repository-level bug localization aims to identify suspected bug locations
	from large codebases, where the main challenge lies in effectively associating natural language
	problem descriptions with appropriate code elements [43]. Meanwhile, repair ingredient code
	extraction aims to obtain hierarchical code contexts related to the bug locations, which typically
	span files, functions and statements, where the key difficulty is cross-hierarchical and multi-dependency
	relationship reasoning, as well as the control of context length [40, 215].

	Chen et al.[40] propose RLCE, a static analysis approach designed to accurately extract
	simplified context from a codebase based on known bug locations. The method first parses the repository
	file and constructs a project structure tree comprising multiple hierarchical layers:
	directories, files, classes, functions, and global variables, and tracks cross-file references,
	such as function calls or class usages, to annotate file nodes with inter-file dependencies.
	RLCE then searches in the project structure tree based on the bug locations (one or more lines of
	code) to obtain the required code segments, including the Error Invoking Function (EIF), its
	caller, the Error Function (EF), and its caller. Finally, these function bodies, function signatures
	with summaries, or call line context slices will be input into prompts for LLMs to perform the
	repair.

	Chang et al.[32] find that existing approaches, although most adopt hierarchical structures,
	neglect the differences in semantic content and dependencies at different levels. For example,
	function-level code relies on function calls, while statement-level code relies on data and control
	dependencies between statements. To address this limitation, they propose BugCerberus, which
	accomplishes the task of targeted bug localization by customized fine-tuning LLMs in three
	levels: file, function, and statement.

	Chen et al.[43] further advanced this line of research by introducing LocAgent, a specialized
	framework for LLM-driven code localization, including bug localization. LocAgent first parses
	the codebase to build heterogeneous directed graphs, where nodes represent code entities at varying
	levels (e.g., files, functions, classes), and edges encode diverse dependency types (e.g., containment,
	invocation, inheritance). These graphs are paired with sparse indexing mechanisms. The framework
	equips an LLM agent with three retrieval tools—SearchEntity, TraverseGraph, and RetrieveEntity- allowing
	it to iteratively search, infer, and refine its understanding of the code structure.
	Experimental results on SWE-bench-Lite demonstrate that LocAgent significantly surpasses
	Agentless [215], SWE-Agent [234], and OpenHands [193] in the bug localization task, especially in
	fine-grained function localization.
	\color{black}

	\begin{tcolorbox}
		[commentbox,title=Reviewer \#2 - Comment 6] Figure 3, Figure 11. To get more information to the
		readers, authors should add the distributions (maybe also references in addition) of each
		category in this figure. The same is true for Figure 11 (add percentage).
	\end{tcolorbox}

	\noindent
	\textbf{Response.}
	% 感谢您的有价值的建议。我们已经分别在图n和图x中添加了研究数量占比情况,这将有利于读者更加直观的xxx。
	% 值得注意的是,尽管原始的图3已经完全被删除,我们仍然在新的图3中加入了百分比的元素。我们只在Task Style和LLM Utilization Tech.两个维度中加入了百分比,因为其他两个维度存统计百分比没有意义,例如目前iterative的推理方式要比agent的工作多不少,展示百分比可能会误导读者忽视agent推理方式,但是这仅仅是因为受限于过去的技术发展。再比如RAG和In-Context Learinig并非二元对立的两种方法,他们互相交叉和促进,因此统计百分比也无意义。总之,我们参考审稿意见的同时,充分考虑细节,对现有的图3和图n做出了优化。

	\begin{figure}[t!]
		\centering
		\includegraphics[width=1\textwidth]{fig/RQ1-v1.pdf}
		\caption{Four discussion dimensions for RQ1.}
		\label{f:rq1}
	\end{figure}

	Thank you for your valuable suggestions. We have incorporated the percentage distribution of
	research quantities into Figure \ref{f:rq1} (
	\color{red}
	Figure 3
	\color{black}
	in manuscript) and Figure \ref{f:flow} (
	\color{red}
	Figure 12
	\color{black}
	in manuscript), which helps readers more clearly and intuitively understand the distribution and
	proportion of research in each category.

	Notably, although the original Figure 3 has been entirely removed, percentage elements have still
	been integrated into the newly designed
	\color{red}
	Figure 3
	\color{black}
	. Specifically, we added percentage annotations for two dimensions: Task Style and LLM Utilization
	Technique. In contrast, percentages were intentionally omitted from the other two dimensions,
	where such representations could be misleading or uninformative.

	For instance, the current dominance of iterative reasoning over agent-based reasoning is largely
	a reflection of historical technological limitations rather than intrinsic superiority.
	Including percentages in this case might unintentionally downplay the value and potential of
	agent-based approaches. Likewise, RAG and In-Context Learning are not mutually exclusive
	paradigms—they overlap and complement one another—rendering percentage-based comparisons
	inappropriate.

	In summary, while carefully incorporating the reviewers’ feedback, we have thoroughly revised
	and optimized the new
	\color{red}
	Figure 3
	\color{black}
	and
	\color{red}
	Figure 12
	\color{black}
	to ensure both clarity and accuracy in the presentation of information.

	\begin{figure}[t!]
		\centering
		\includegraphics[width=0.8\textwidth]{fig/chart-1.pdf}
		\caption{The flow of APR-specific and General information types to patch generation
		techniques.}
		\label{f:flow}
	\end{figure}

	\begin{tcolorbox}
		[commentbox,title=Reviewer \#2 - Comment 7] Line 32, Page 23 (section 5.3.2): authors should
		add a table/figure that helps better describe Training Strategies
	\end{tcolorbox}

	\noindent
	\textbf{Response.}

	Thank you very much for your valuable suggestion and reminder. We agree that summarizing the training
	strategies in tabular form is necessary, especially considering that we have added new content
	to this section in the previous round of revisions.

	The summary and comparison of training strategies are now presented in
	\color{red}
	Table 6
	\color{black}
	. We have carefully added this table to provide a structured overview of each training strategy
	category, including their descriptions, representative works, input/output, and application
	scenarios. This addition is intended to help readers more intuitively and comprehensively understand
	the distinctions among different training strategies.

	\begin{tcolorbox}
		[commentbox,title=Reviewer \#2 - Comment 8] Line 41, Page 26 (section 7): I appreciate the
		takeaway section that summarized this survey. I suggest the authors connect the takeaways to
		the RQs. Besides, I also recommend authors organize the takeaways in a graph or table instead
		of plain text.
	\end{tcolorbox}

	% Takeaway的展示问题已经在上一轮修改中解决
	\noindent
	\textbf{Response.}

	The issue with presenting the Takeaways clustered in one whole section has been well resolved in
	the previous round of revisions.

	\bibliographystyle{ACM-Reference-Format}
	\bibliography{reference}
\end{document}
